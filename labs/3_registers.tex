

\chapter{Хранение и передача данных}

Хранить данные очень просто. Например, можно записать несколько единичек
и ноликов на бумаге, а потом, если нужно это вспомнить, прочитать то, что написано.

Если нужно создать какое-то автоматическое устройство, бумажка уже плохо
подходит. Можно сохранить информацию в состоянии тумблеров. Переключить
их, <<записав>> таким образом какую-то последовательность в виде
состояния переключателей, а не в виде символов на бумаге. Если эти тумблеры
включить в какие-то электрические цепи, можно даже устроить <<считывание>>
этой информации. Например, включится мотор, зажжётся лампочка или
отправится электронное письмо из этих символов.

Но что если у нас нет рядом лишнего человека, который будет переключать
все эти тумблеры? Хотелось бы, чтобы при появлении в некоторой <<входной>>
цепи сигнала (это будет операция записи) можно было бы затем получать его же
в <<выходной>> цепи (это уже операция чтения) через произвольное время,
даже если входного сигнала уже нет.

Эти свойства описывают ячейку памяти. Они могут отличаться количеством
хранимой информации, способами управления и элементной базой.
Мы сейчас рассмотрим пару вариантов, которые удобно реализовывать
с помощью электромагнитных реле.

\section{RS-триггер}

Триггер --- это простейшее устройство для хранения одного бита информации.
Если он включен, там хранится единица, а если выключен --- хранится ноль.

Хранение одного бита можно сделать с помощью одного реле.
Если оно включается с помощью внешнего сигнала, то реле <<сохраняет>> единицу.
Но когда внешний сигнал пропадает, сохранённая единица пропадает тоже,
потому что на катушке реле больше нет напряжения.

Чтобы включённое состояние запоминалось, нужно организовать обратную
связь через один из контактов реле. Как только реле включается с помощью
входа $SET$, на его
обмотку подаётся напряжение через его же замкнутый контакт. Поэтому даже
если перестать подавать напряжение на вход $SET$, реле всё равно останется включённым:

\begin{center}
\includegraphics{schemes/trigger1.png}
\end{center}

Но если такое реле уже включилось, выключить его не получится. Единственный способ~---
это отключить питание всей схемы. Тогда контакты разомкнутся и при повторной подаче питания реле будет
в выключенном состоянии.

Чтобы не отключать всю схему, нужно предусмотреть отдельное реле, позволяющее
отключить питание только для запоминающего реле:

\begin{center}
\includegraphics{schemes/trigger2.png}
\end{center}

Теперь подав сигнал на вход $RESET$, можно сбросить наше реле-триггер. И там опять окажется ноль.

Триггер с двумя входами $RESET$ и $SET$ называется RS-триггером.
На практике удобно объединить несколько таких триггеров в многобитовый регистр
и использовать для них единый сигнал сброса, сэкономив несколько реле:

\begin{center}
\includegraphics{schemes/trigger3.png}
\end{center}

\section{Шина}

Шина --- это набор проводников, объединённых общим назначением.
К этим проводникам подключаются несколько устройств для обмена сигналами.

Например, если к внутренним шинам компьютера могут быть
подключены память и жёсткий диск. Тогда по шине данных передаются данные между
процессором и памятью или процессором и жёстким диском.
Шина адреса используется для выбора определённой ячейки в памяти.
А по шине управления приходят сигналы, позволяющие отличать обращение к памяти
от обращения к диску.

Так как к шине обычно подключается больше одного входа и больше одного выхода,
необходимо изолировать лишние устройства, когда шина используется другими.
Например, когда процессор взаимодействует с памятью через шины
данных или адреса, жёсткий диск не должен подключаться к тем же шинами.
В противном случае в сигналах будет путаница вплоть до того, что одно из
устройств сгорит.

В релейном компьютере мы можем использовать для подключения к четырёхбитной шине
шинный формирователь из одного реле:

\begin{center}
\includegraphics{schemes/bus.png}
\end{center}

Когда такое реле включается, оно может соединить какое-то устройство (например, регистр)
с шиной. Подключение будет двунаправленным, поэтому с помощью этой схемы можно как
записывать данные в регистр, так и считывать оттуда.

Шинные формирователи подключаются к каждому устройству, поэтому
для передачи данных и вычислений требуется формировать множество сигналов $SELECT$.

\section{Шины в конструкторе}

В конструкторе все модули могут подключаться к шинам по которым передаются от~$1$ до~$4$ сигналов.

Например, $4$ сигнала могут использоваться для управления модулем регистра.
Или для передачи данных из одного регистра в другой.
Каждым из таких сигналов можно управлять с помощью модуля переключателей,
соединив его с соответствующим разъёмом регистрового модуля.

\section{Регистр}

Регистры внутри процессоров (или периферийных устройств)
используются для хранения данных и для проведения вычислений.
Регистры могут иметь выделенное назначение (например, хранить режим работы устройства
или счётчик инструкций)
или же используются почти для всего (адрес, операнд арифметических операций,
значение для ввода-вывода).

На схеме регистрового модуля можно увидеть $8$ реле:

\begin{landscape}
\begin{center}
\includegraphics[width=\linewidth]{schemes_components/register.png}
\end{center}
\end{landscape}

Четыре из них это реле-триггеры,
ещё одно реле нужно для обнуления регистра (оно расположено на схеме слева)
а три оставшиеся используются для подключения к шинам данных (эти реле справа).

Для хранения данных используются RS-триггеры, описанные выше,
но сигнал для сброса у них общий. Таким образом, записывать
данные можно только во все четыре бита сразу. Поэтому такой
модуль и реализует функции цельного регистра, а не нескольких
разрозненных RS-триггеров.

Также модуль регистра содержит три шинных формирователя.
Их можно использовать для подключения триггеров к разным устройствам
через три шины. Например, одна шина может быть предназначена
для первого операнда при вычислениях, вторая для второго операнда,
а третья для копирования данных между регистрами.

Выглядит плата регистрового модуля так:

\begin{center}
\includegraphics{boards/register.png}
\end{center}

Если на плату регистра установить только реле шинных формирователей,
получится модуль, который может подключать четырёхбитные сигналы
(полученные из разъёма для прямого чтения значений триггеров) к одной
из выбранных шин. Раньше мы уже использовали такой вариант модуля
для демонстрации работы тумблеров.

На плате регистра есть следующие разъёмы:
\begin{itemize}
  \item Слева и справа: управляющие сигналы сброса и выборки.
        Можно подключить тумблеры
        для ручного включения сигналов. Также можно соединить несколько
        модулей регистра, чтобы управлять одним набором сигналов сразу
        для $8$, $12$ \ldots бит.
  \item Сверху и снизу: три шины данных. Реле регистра могут
        подключаться к шинам для записи или чтения данных.
  \item Дополнительные разъёмы с битами $0-3$ и $2-3$ для чтения или
        записи значения без подключения к шине.
\end{itemize}

Чтобы прочитать значение регистра нужно активировать сигнал выборки
его на нужную шину. А уже через шину сигналы поступят на вход какой-то
другой схемы.

Для записи значения в регистр нужно сначала обнулить его триггеры,
иначе запись не произойдёт в тех битах, где уже хранятся единицы.
После этого регистр подключается к нужной шине, и триггеры
защёлкивают нужное значение.

\subsection{Практикум}


Список модулей:
\begin{itemize}
    \item Модуль переключателей: $2$ штуки
    \item Регистровый модуль: $1$ штука
\end{itemize}

Протестировать работу регистра можно собрав следующую схему:

\includegraphics[width=0.5\columnwidth]{photo/register.jpg}

\begin{itemize}
  \item Тумблеры слева управляют работой регистра. Бит~$0$ --- сброс триггеров регистра,
        бит~$1$ --- выборка на шину~$1$.
        Тумблеры~$2$ и $3$ здесь не используются.
  \item Тумблеры сверху нужны для ввода значения регистра. Когда он подключается к шине~$1$,
        значения, набранное на тумблерах, записывается в регистр.
\end{itemize}

\subsubsection{Регистр без шины}

Сначала немного изменим схему, чтобы продемонстрировать, зачем нужно подключение регистра к шине.

\includegraphics[width=0.5\columnwidth]{photo/register_direct.jpg}

\begin{enumerate}
    \item Подключить тумблеры проводом (для этого разъёма нет другого варианта) к битам $0-3$ вместо шины.
          Это позволит нам управлять включением триггерных реле непосредственно с помощью тумблеров.
    \item Набирать значение на тумблерах, подключенных проводом.
          Убедиться, что биты переключаются в $1$, но не возвращаются в $0$, если тумблер переключить обратно.
    \item Обнулить тумблеры, управляющие триггерами (те, что подключены проводом).
    \item Включить и выключить сигнал сброса (тумблеры, подключенные к боковому разъёму).
          Убедиться, что значения всех битов теперь $0$.
\end{enumerate}

\subsubsection{Регистр с шиной}

Теперь можно вернуть схему в исходное состояние. Ведь неудобство изменённого варианта
в том, что приходится выключать все тумблеры, чтобы можно было просто сбросить все биты регистра.
А ведь это могут быть не тумблеры, а какое-то другое устройство (к примеру, ещё один
регистр), который может быть не согласен сбрасываться.

\begin{enumerate}
    \item Отключить все управляющие сигналы.
    \item Набрать значение на тумблерах для данных. Убедиться, что это не влияет на регистр.
    \item Включить и выключить сигнал выборки на шину $1$. Убедиться, что данные записались в регистр.
    \item Включить и выключить сигнал сброса. Убедиться, что значения всех битов теперь $0$.
          Состояния триггеров, подключенных к шине, никак не помешали выполнить сброс.
\end{enumerate}

\subsection{Задачи}

\begin{enumerate}
    \item Возьмите два регистровых модуля и придумайте,
          как скопировать данные из одного регистра в другой, не запоминая их в голове.
\end{enumerate}


\section{Шина и регистровый файл}

Несколько регистров можно соединить в регистровый файл.
Регистровый файл --- это набор регистров, которые принадлежат одному устройству.
Например, в регистровом файле процессора будет счётчик инструкций,
регистры для хранения арифметических операндов, а также для хранения адреса
при обращении к памяти.

Регистры объединены в регистровый файл, потому что процессор при выполнении операций
делает с регистрами похожие действия. Копирует один в другой, берёт их в разном порядке
для вычислений, записывает в память. Поэтому все регистры должны находиться <<рядом>>,
чтобы схема работы с ними не была слишком сложной.

В нашем конструкторе у каждого из регистров есть сигналы выборки на одну из трёх шин.
Для одного регистра эти сигналы могут активироваться как поочерёдно, так и одновременно
(хотя второе требуется редко).
Если два регистра подключены к одной шине одновременно,
то значения одного копируются в другой. Если точнее,
включённые биты включают аналогичные в другом регистре, то есть
копирование происходит в обе стороны одновременно.

Нулевые биты при этом копироваться не могут. Для записи нулей
регистр необходимо сбросить.


\subsection{Практикум}


Список модулей:
\begin{itemize}
    \item Модуль переключателей: $5$ штук
    \item Регистровый модуль: $4$ штуки
\end{itemize}

\includegraphics[width=\columnwidth]{photo/register_file.jpg}

Запись в регистры:

\begin{enumerate}
    \item Отключить все управляющие сигналы.
    \item Набрать значение на тумблерах, подключённых к шине данных $1$.
    \item Подключить с помощью тумблера регистр к шине $1$. Убедиться, что в него записалось набранное значение.
    \item Отключить регистр от шины.
    \item Установить другое положение тумблеров, подключённых к шине данных.
    \item Подключить любой другой регистр к шине $1$. Убедиться, что в него записалось набранное значение.
\end{enumerate}

Копирование значения:

\begin{enumerate}
    \item Отключить все управляющие сигналы.
    \item Подключить регистр с ненулевыми битами к шине $2$.
    \item Подключить пустой регистр к шине $2$. убедиться, что он получил такое же значение, что и в первом регистре.
    \item Отключить все управляющие сигналы.
    \item Аналогично проверить шину $3$.
\end{enumerate}


\subsection{Задачи}

\begin{enumerate}
    \item Пронумеруйте в уме регистры слева направо или сверху вниз (смотря как у вас лежит плата на столе).
          Теперь сделайте так, чтобы в регистр $1$ попало значение $0001$, в регистр $2$ значение $0010$,
          и так далее.
\end{enumerate}
