\chapter{Релейный конструктор}

\section{Модули}

Релейный конструктор включает не просто реле и провода, как можно было бы подумать,
а несколько видов готовых модулей, выполняющих простые функции.
Каждый модуль состоит из печатной платы, нескольких разъёмов на ней
и других компонентов, в зависимости от его функций (реле, переключатели, диоды, резисторы).

Это похоже на простые микросхемы: один модуль может складывать числа, другой
хранить несколько битов информации, и так далее. Ещё есть модуль с переключателями,
чтобы человек мог управлять схемой.

В следующих главах мы будем постепенно знакомиться с каждым из модулей, соединять их друг с другом,
чтобы делать вычислительные устройства: несложные логические схемы, игрушки, калькулятор и даже компьютер.

\section{Соединительные разъёмы}

Модули соединяются между собой с помощью разъёмов по краям плат или плоских кабелей.
Часто они расположены парами, чтобы можно было либо стыковать платы, либо пользоваться кабелем.
В каждом разъёме для подключения платы или кабеля есть контакты для питающего напряжения
(контакт~$1$) и нуля (контакт~$8$).
Поэтому все разъёмы электрически совместимы друг с другом, но логически их соединение не всегда
имеет смысл.

Кроме питающего напряжения через соединения передаются и полезные сигналы, занимающие до четырёх контактов
(номера от~$2$ до~$5$), в зависимости от назначения модуля.
Они используются для кодирования четырёхбитного числа или передачи четырёх
управляющих сигналов разного назначения.

Контакты $6$ и $7$ в текущей версии конструктора не используются.

\chapter{Первые эксперименты с конструктором}


\section{Модуль переключателей}

\begin{center}
\includegraphics{boards/switches.png}
\end{center}

Модуль с переключателями (или тумблерами) используется для установки
уровней $0$ или $1$ на выходном разъёме вручную.
Присоединяя его к разным разъёмам, можно задавать четырёхбитное число,
либо переключать до четырёх управляющих сигналов.

Проще всего проверить работу тумблеров, подключив их к управляющей шине
регистрового модуля, в который вставлены только $4$ реле.

Входных разъёмов у этого модуля нет, только выходные, на которые попадают сигналы
с тумблеров. Все разъёмы подключены параллельно, поэтому можно синхронно управлять
одним набором переключателей сразу несколькими модулями, если подключить их все.

\subsection{Практикум}


Список модулей:
\begin{itemize}
    \item Модуль переключателей: $1$ штука
    \item Регистровый модуль как шинный формирователь: $1$ штука
\end{itemize}

\includegraphics[width=0.5\columnwidth]{photo/switches.jpg}

\begin{enumerate}
    \item Переключать тумблеры. Убедиться, что положение одного тумблера меняет состояние одного реле.
    \item Запомнить включенное и выключенное состояния тумблеров, а также их порядок по отношению к реле,
          чтобы позднее не было проблем с управлением другими схемами.
\end{enumerate}

\subsection{Задачи}

\begin{enumerate}
    \item Придумать, как соединить тумблеры для выполнения логического ИЛИ.
    \item Придумать, как соединить тумблеры для выполнения логического И.
\end{enumerate}
