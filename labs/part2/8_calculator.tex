\chapter{Калькулятор}

\section{Калькулятор для 7 операций}

Все вычислители можно соединить с одним и тем же регистром, как хранилищем результата, чтобы получить простейший
калькулятор. Доступны следующие операции:
\begin{enumerate}
    \item Инверсия
    \item Сдвиг вправо
    \item Побитовое И
    \item Побитовое ИЛИ
    \item Исключающее ИЛИ
    \item Сложение
    \item Вычитание
\end{enumerate}

\section{Практикум}

Список модулей:
\begin{itemize}
    \item Модуль переключателей: $10$ штук
    \item Модуль унарных операций: $2$ штуки
    \item Сумматор: $2$ штуки
    \item Модуль логических операций: $1$ штука
    \item Регистровый модуль как шинный формирователь: $2$ штуки
    \item Регистровый модуль: $1$ штука
    \item Модуль шины: $1$ штука
    \item Соединительные шлейфы: $4$ штуки
\end{itemize}

Уже проверенные модули для разных операций соединяются вместе.
Для этого используются несколько шинных формирователей
на платах регистров. На этих платах не установлены реле для хранения
битов. Вместо этого шины подключаются к одному и тому же регистру.
Так в него можно записывать любой из $7$ результатов вычислений.

\includegraphics[width=\columnwidth]{photo/calculator.jpg}

Любую из операций можно выполнить так:

\begin{enumerate}
    \item Отключить все управляющие сигналы.
    \item Набрать входные данные для нужной операции.
    \item Подключить выход нужного модуля к регистру тумблером.
    \item Наблюдать результаты вычислений.
    \item Сбросить значение регистра.
\end{enumerate}


\chapter{Расширение вычислений до восьми бит}

У регистров и вычислительных модулей справа и слева есть разъёмы для расширения разрядности.

Для регистров через эти разъёмы передаются сигналы сброса и подключения к шинам. Поэтому
один набор тумблеров может использоваться для двух четырёхбитных плат-регистров, если они
соединены в один восьмибитный регистр.

Для вычислительных модулей через боковые разъёмы передаются сигналы переноса (в случае сдвига и сложения).

\section{Практикум}

% TODO: запись восьмибитного значения в регистр

Соединить два регистра и два сумматора. Шина $3$ регистров подключается к сумматору.
У правого (младшего) сумматора входящий перенос перемычкой устанавливается в $0$.
У левого (старшего) сумматора перемычка для переноса убирается, потому что
сигнал переноса приходит от младших битов (из правого сумматора).

\includegraphics[width=0.5\columnwidth]{photo/8bit.jpg}

\begin{enumerate}
    \item Отключить все управляющие сигналы.
    \item Набрать входные данные $0011 1001$ и $0010 1001$.
    \item Подключить выход сумматора к регистру тумблером $D3$.
    \item Наблюдать результат вычислений $0110 0010$.
\end{enumerate}

