\chapter{Релейный конструктор}

\section{Модули}

Релейный конструктор включает не просто реле и провода, как можно было бы подумать,
а несколько видов готовых модулей, выполняющих простые функции.
Каждый модуль состоит из печатной платы, нескольких разъёмов на ней
и других компонентов, в зависимости от его функций (реле, переключатели, диоды, резисторы).

Это похоже на простые микросхемы: один модуль может складывать числа, другой
хранить несколько битов информации, и так далее. Ещё есть модуль с переключателями,
чтобы человек мог управлять схемой.

В следующих главах мы будем постепенно знакомиться с каждым из модулей, соединять их друг с другом,
чтобы делать вычислительные устройства: несложные логические схемы, игрушки, калькулятор и даже компьютер.

\section{Соединительные разъёмы}

Модули соединяются между собой с помощью разъёмов по краям плат или плоских кабелей.
Часто они расположены парами, чтобы можно было либо стыковать платы, либо пользоваться кабелем.
В каждом разъёме для подключения платы или кабеля есть контакты для питающего напряжения
(контакт~$1$) и нуля (контакт~$8$).
Поэтому все разъёмы электрически совместимы друг с другом, но логически их соединение не всегда
имеет смысл.

Но если какой-то разъём не используется для передачи данных,
к нему можно подключить источник питания. А остальные модули запитаются за счёт того,
что контакты питающего напряжения и нуля есть во всех разъёмах.

Кроме питающего напряжения через соединения передаются и полезные сигналы, занимающие до четырёх контактов
(номера от~$2$ до~$5$), в зависимости от назначения модуля.
Они используются для кодирования четырёхбитного числа или передачи четырёх
управляющих сигналов разного назначения.

Контакты $6$ и $7$ в текущей версии конструктора не используются.

\chapter{Первые эксперименты с конструктором. Модуль переключателей}


Модуль с переключателями (или тумблерами) используется для установки
уровней $0$ или $1$ на выходном разъёме вручную.
Присоединяя его к разным разъёмам, можно задавать четырёхбитное число,
либо переключать до четырёх управляющих сигналов.

\begin{center}
\includegraphics{schemes_components/switches.png}
\end{center}

Реализована эта схема в виде такой печатной платы:

\begin{center}
\includegraphics{boards/switches.png}
\end{center}

Входных разъёмов у модуля с тумблерами нет, ведь никакие данные в него не поступают,
чтобы переключить реле или зажечь светодиод. Ведь на плате ничего этого нет.
Поэтому у модуля есть только выходные разъёмы, на которые попадают сигналы
с тумблеров. Все разъёмы подключены параллельно, поэтому можно синхронно управлять
одним набором переключателей сразу несколькими модулями, если подключить их все.


\section{Практикум}

Проще всего проверить работу тумблеров, подключив их к модулю унарных логических операций.

Список модулей:
\begin{itemize}
    \item Модуль переключателей: $1$ штука
    \item Модуль унарных логических операций: $1$ штука
    \item Проводной шлейф для соединения модулей: $1$ штука
\end{itemize}

Сначала соберём следующую схему.

\includegraphics[width=0.5\columnwidth]{photo/switches.jpg}

Источник напряжения можно подключить к любому незадействованному разъёму, потому что на каждом
из них есть контакты для питания схемы.

Обычно подключать источник питания нужно последним, но если нужно переставить какой-то
модуль в другое место или добавить какой-то проводник, то можно это делать
и при подключённом питании. Ничего не сломается.

\begin{enumerate}
    \item Переключать тумблеры. Убедиться, что положение одного тумблера меняет состояние одного реле.
    \item Запомнить включенное и выключенное состояния тумблеров, а также их порядок по отношению к реле,
          чтобы позднее не было проблем с управлением другими схемами.
\end{enumerate}

Как видите, реле переключаются, в зависимости от положений тумблеров.
При этом они генерируют выходной сигнал, но с ним мы поработаем позднее.

Теперь заменим прямое соединение двух плат на соединение с помощью провода.

\includegraphics[width=0.5\columnwidth]{photo/switches2.jpg}

Попереключайте тумблеры как раньше, чтобы убедиться, что всё работает ровно так же.
То есть, использованные способы соединения модулей эквивалентны.

\section{Задачи}

\begin{enumerate}
    \item Придумать, как соединить два модуля с тумблерами для выполнения логического ИЛИ.
    % \item Придумать, как соединить тумблеры для выполнения логического И.
\end{enumerate}
