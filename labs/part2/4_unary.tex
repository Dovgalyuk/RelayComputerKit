\chapter{Унарные логические операции}

% Таблицы истинности

\section{Отрицание}

Одна из самых простых операций, которая используется в современных
компьютерах --- это операция отрицания. Она преобразует единицу
в ноль, а ноль в единицу.

Такую логику можно реализовать с помощью нормально замкнутого контакта реле:

\begin{center}
\includegraphics{schemes/not.png}
\end{center}

Когда реле выключено, ток может течь через контакт, поэтому на выходе $B$
получается единица. Если же на входе $A$ появляется единица, реле переключается
и ток через нормально замкнутый контакт не течёт. На выходе $B$ теперь ноль.

\section{Сдвиг}

Операция сдвига ещё более простая по сравнению с отрицанием.
Суть её в том, что биты в числе перемещаются по какому-либо
принципу (влево, вправо, по кругу).

В конструкторе реализован сдвиг вправо.
Нулевой бит вытесняется из числа (и может использоваться отдельно).
Первый бит становится нулевым, второй первым и так далее.
Такой сдвиг эквивалентен делению на два, поэтому деление на степень двойки в программах
процессор вычисляет с помощью инструкции сдвига.

Для реализации сдвига вправо можно даже не использовать реле:

\begin{center}
\includegraphics{schemes/shr.png}
\end{center}

В конструкторе есть готовый модуль только для сдвига вправо. Ещё в вычислениях часто используется
сдвиг влево (эквивалентный умножению на два), но его легко реализовать с помощью сложения числа с самим собой.

\section{Модуль унарных логических операций}

Модуль для унарных операций выполняет действия над $4$-битным числом: сдвиг вправо и инверсия битов.
Эти действия выполняются <<одновременно>>. То есть с одного выхода можно прочитать сдвинутое число,
а с другого инвертированное. И одновременно записать это в разные регистры.

\begin{center}
\includegraphics[width=\columnwidth]{schemes_components/logic_unary.png}
\end{center}


\begin{center}
\includegraphics{boards/logic_unary.png}
\end{center}


Модуль для унарных логических операций имеет следующие разъёмы:
\begin{itemize}
  \item Слева и справа: шины для каскадирования нескольких модулей.
        Можно использовать, когда нужен сдвиг чисел больше четырёх бит.
  \item Сверху: разъёмы для подключения сдвигаемого операнда.
        Можно подсоединить к какой-нибудь шине, а можно к модулю переключателей или регистру.
  \item Снизу: два разъёма с выходными сигналами. На один поступает инвертированный
        операнд, а на другой сдвинутое вправо значение операнда.
\end{itemize}


\section{Практикум}


Список модулей:
\begin{itemize}
    \item Модуль переключателей: $2$ штуки
    \item Модуль унарных операций: $1$ штука
    \item Регистровый модуль: $1$ штука
\end{itemize}


На вход модуля унарных операций подключается модуль с тумблерами.
Выходы подключаются к шинам регистра.


\includegraphics[width=0.5\columnwidth]{photo/unary.jpg}

\begin{enumerate}
    \item Отключить все управляющие сигналы.
    \item Набрать на тумблерах со входными данными значение $1100$.
    \item Подключить выходной регистр к шине $1$. Убедиться, что в него записалось значение $0011$ (инверсия).
    \item Отключить регистр от шины, сбросить его значение.
    \item Подключить выходной регистр к шине $2$. Убедиться, что в него записалось значение $0110$ (сдвиг вправо).
\end{enumerate}

\section{Задачи}

\begin{enumerate}
    \item Собрать устройство для одновременного вычисления сдвига и инверсии. По сигналу с одного тумблера
          эти оба результата должны записаться в два разных регистра.
    \item Собрать устройство, позволяющее сдвигать содержимое регистра и записывать результат обратно.
          Занести в регистр значение $1000$ и сдвигать его до тех пор, пока не получится $0001$.
          Придётся использовать два регистровых модуля: с одного считывается операнд, а в другой записывается результат.
          Вторую часть можно выполнять на скорость.
\end{enumerate}

