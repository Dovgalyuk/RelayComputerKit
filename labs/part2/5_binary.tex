\chapter{Бинарные логические операции}

\section{Логическое И}

Реле представляет собой управляемый выключатель или переключатель.
Если на один контакт $A$ выключателя подать сигнал, то на другом $C$ он появится
только если на обмотке реле $B$ есть напряжение (логическая единица).

\begin{center}
\includegraphics{schemes/and.png}
\end{center}

Получается, что на выходе выключателя сигнал будет только тогда, когда
реле включено и на входе выключателя тоже не ноль. Это означает, что такая
схема реализует операцию логическое <<И>>: $C = A \land B$.



\section{Логическое ИЛИ}


Схему для выполнения логического <<ИЛИ>> можно получить, если соединить два выхода
(две цепи). Тогда напряжение на объединённом выходе появится в любом
из вариантов, если на первом выходе единица или на втором: $C = A \lor B$.

\begin{center}
\includegraphics{schemes/or1.png}
\end{center}

Но у этого способа есть один недостаток. Если на входе $A$ окажется единица,
то так как все входы и выходы соединены в одну цепь, поданное через $A$ напряжение
будет влиять и на вход $B$. И когда $B$ используется ещё где-то,
всё заработает неправильно, какое-то реле включится.
Такой вот паразитный сигнал --- $A$ влияет на выходы, зависящие от $B$.

Поэтому иногда стоит использовать схему с реле для логического <<ИЛИ>>:

\begin{center}
\includegraphics{schemes/or2.png}
\end{center}


\section{Исключающее ИЛИ}

Схема для вычисления <<исключающего или>> несколько сложнее.
Результат операции должен равняться единице только тогда,
когда операнды не равны. То есть если на одном из входов уже
было напряжение, а потом оно появляется и на другом входе,
выход из состояния единицы должен переключиться в ноль.

Такую логику проще всего реализовать с помощью двух переключающих контактов:

\begin{center}
\includegraphics{schemes/xor.png}
\end{center}

Благодаря перекрёстному соединению переключателей проводниками,
сигнал на выходе $C$ появляется только тогда, когда переключатели на реле
находятся в противоположных положениях.

\section{Модуль бинарных логических операций}

\begin{center}
\includegraphics[width=\columnwidth]{schemes_components/logic_binary.png}
\end{center}


\begin{center}
\includegraphics{boards/logic_binary.png}
\end{center}

Модуль логических операций поддерживает вычисление <<И>>, <<ИЛИ>> и <<Исключающее ИЛИ>>.
У модуля есть два четырёхбитных входа и три выхода.
Каждый выход отвечает за одну операцию.

Модуль имеет следующие разъёмы:
\begin{itemize}
  \item Слева и справа: шины для каскадирования нескольких модулей.
        Полезных сигналов через них не передаётся (в отличие от такого же разъёма у модуля унарных операций),
        но можно их использовать для фиксации нескольких логических модулей вместе,
        когда нужны вычисления разрядностью больше четырёх бит.
  \item Сверху: разъёмы для подключения сигналов-операндов.
        Можно подсоединить к шинам, а можно к модулям переключателей или регистрам.
  \item Снизу: три разъёма с выходными сигналами: результатами вычисления
        операций <<И>>, <<ИЛИ>>, <<Исключающее ИЛИ>>.
\end{itemize}


% \subsection{Подготовка}

% \begin{enumerate}
%     \item Как могла бы выглядеть схема для выполнения операцию XOR?
% \end{enumerate}

\section{Практикум}

\subsection{Работа модуля}

Список модулей:
\begin{itemize}
    \item Модуль переключателей: $3$ штуки
    \item Модуль логических операций: $1$ штука
    \item Регистровый модуль: $1$ штука
\end{itemize}


Ко входам модуля подключаются два модуля с тумблерами, а к выходам --- регистр,
куда будет записываться результат.


\includegraphics[width=0.5\columnwidth]{photo/logic.jpg}

\begin{enumerate}
    \item Отключить все управляющие сигналы.
    \item Набрать на тумблерах первого операнда значение $1100$.
    \item Набрать на тумблерах второго операнда значение $1010$.
    \item Подключить выходной регистр к шине $1$. Убедиться, что в него записалось значение $1000$ (AND).
    \item Отключить регистр от шины, сбросить его значение.
    \item Подключить выходной регистр к шине $2$. Убедиться, что в него записалось значение $1110$ (OR).
    \item Отключить регистр от шины, сбросить его значение.
    \item Подключить выходной регистр к шине $3$. Убедиться, что в него записалось значение $0110$ (XOR).
\end{enumerate}


\subsection{Управление лампочкой с помощью нескольких выключателей}

Управлять одной лампочкой может быть очень полезным, если эта лампочка освещает
длинный коридор (конечно, в коридоре лампочек будет несколько, но обычно
они все зажигаются одновременно). Например, человеку может потребоваться включить
свет, заходя в коридор с одной стороны, а выключить, выходя с другой.
Эту задачу можно решить с помощью механических переключателей (у которых один вход и два выхода),
но мы попробуем собрать аналогичную логическую схему используя только выключатели
(вход один и выход тоже один).

В качестве лампочки будем использовать одно из управляющих реле регистрового модуля.
Схема должна позволять в любой момент зажечь или выключить лампочку переключением любого
из трёх тумблеров.

Построим таблицу истинности для схемы из выключателей и лампочки: входами $A$, $B$, $C$ будут
положения тумблеров, а выходом $R$ --- состояние лампочки.

Сначала всё выключена, поэтому лампочка тоже не горит:

\begin{center}
\begin{tabular}{|c|c|c|c|}
 \hline
 \textbf{A} & \textbf{B} & \textbf{C} & \textbf{R} \\ \hline
 $0$ & $0$ & $0$ & $0$ \\ \hline
\end{tabular}
\end{center}

Если один из тумблеров переключить, лампочка загорится:

\begin{center}
\begin{tabular}{|c|c|c|c|}
 \hline
 \textbf{A} & \textbf{B} & \textbf{C} & \textbf{R} \\ \hline
 $0$ & $0$ & $0$ & $0$ \\ \hline
 $1$ & $0$ & $0$ & $1$ \\ \hline
 $0$ & $1$ & $0$ & $1$ \\ \hline
 $0$ & $0$ & $1$ & $1$ \\ \hline
\end{tabular}
\end{center}

Потом можно выключить уже нажатый тумблер (и вернуться к первой строке таблицы),
а можно <<дойти до конца коридора>> и переключить другой. Лампочка должна погаснуть:

\begin{center}
\begin{tabular}{|c|c|c|c|}
 \hline
 \textbf{A} & \textbf{B} & \textbf{C} & \textbf{R} \\ \hline
 $0$ & $0$ & $0$ & $0$ \\ \hline
 $1$ & $0$ & $0$ & $1$ \\ \hline
 $0$ & $1$ & $0$ & $1$ \\ \hline
 $0$ & $0$ & $1$ & $1$ \\ \hline
 $1$ & $1$ & $0$ & $0$ \\ \hline
 $1$ & $0$ & $1$ & $0$ \\ \hline
 $0$ & $1$ & $1$ & $0$ \\ \hline
\end{tabular}
\end{center}

И остаётся вариант, когда переключены все три тумблера. Так как число переключений нечётное,
лампочка загорается снова:

\begin{center}
\begin{tabular}{|c|c|c|c|}
 \hline
 \textbf{A} & \textbf{B} & \textbf{C} & \textbf{R} \\ \hline
 $0$ & $0$ & $0$ & $0$ \\ \hline
 $1$ & $0$ & $0$ & $1$ \\ \hline
 $0$ & $1$ & $0$ & $1$ \\ \hline
 $0$ & $0$ & $1$ & $1$ \\ \hline
 $1$ & $1$ & $0$ & $0$ \\ \hline
 $1$ & $0$ & $1$ & $0$ \\ \hline
 $0$ & $1$ & $1$ & $0$ \\ \hline
 $1$ & $1$ & $1$ & $1$ \\ \hline
\end{tabular}
\end{center}

Такая таблица истинности соответствует операции <<исключающее или>>, только для трёх операндов
вместо двух. Можете убедиться самостоятельно, что таблица описывает такое выражение:

$$R = A\ xor\ B\ xor\ C = (A\ xor\ B)\ xor\ C = A\ xor\ (B\ xor\ C)$$

Чтобы его <<посчитать>> с помощью модулей конструктора нужно немного подумать.
Логический модуль считает $xor$ для пары четырёхбитных операндов. А у нас тут есть три однобитных.
То есть нужно либо взять два логических модуля, либо подавать сигнал с выхода одного
однобитного $xor$ на вход другого внутри одного и того же модуля.

Первый вариант можно реализовать с помощью такой схемы:

\includegraphics[width=\columnwidth]{photo/three_switches_xor.jpg}

Тут реализуется формула, приведённая выше: сначала выполняется $A\ xor\ B$,
а потом с полученным результатом делается $xor\ C$.
Каждый из входных битов расположен на своей плате переключателей.
Значение выходного бита показывается с помощью реле на плате регистрового модуля.

Чтобы использовать только один логический модуль, понадобится пара кросс-модулей
для перемещения битов с одного места на другое:

\includegraphics[width=\columnwidth]{photo/three_switches_xor2.jpg}

Здесь входные данные задаются с помощью трёх тумблеров: $0 - A$, $1 - B$, $2 - C$.
Все сигналы с них подаются на первый вход логического модуля. Но сигнал $B$ мы тут не используем,
а с помощью коммутационной матрицы перенаправляем его на нулевой бит второго входа логического модуля.

Получается, что нулевой бит на выходе будет равен $A\ xor\ B$. С помощью второй матрицы
мы передаём выходной сигнал $xor$ на второй бит второго входа. В результате второй бит будет равен
$(A\ xor\ B)\ xor\ C$, а чтобы это увидеть, к выходу $xor$ также подключен ещё и регистровый модуль с одним
управляющим реле. Эти вычисления в реальности происходят последовательно, но мы этого не заметим,
потому что сигналы распространяются со скоростью света, а реле переключается хоть и помедленнее,
но тоже достаточно быстро.

\section{Задачи}

\begin{enumerate}
    \item Собрать устройство для одновременного вычисления всех трёх логических операций. По сигналу с одного тумблера
          результаты должны записаться в три разных регистра.
\end{enumerate}

% -------------------------------------------------------------------------------------------------------
% TODO: низкая наглядность, слишком много соединений для варианта с одной простой коммутационной матрицей
% \subsubsection{Игра <<Кто быстрее>>}

% кнопка для "своей игры"

% Входы ABCD
% Выходы WXYZ

% N = ~W & ~X & ~Y & ~Z
% set W = A & N

% Регистровый модуль для защелкивания результата
% Несколько модулей для формирования правильного порядка сигналов для AND при вычислении N
% Инвертор и AND для сигнала N
% AND для сигналов SET
% Модуль для размножения N
% -------------------------------------------------------------------------------------------------------

% \subsubsection{Игра <<Волк, коза и капуста>>}

% Тумблеры показывают положение волка, козы, капусты и человека.
% Схема показывает ошибку, если позиция запрещённая.
