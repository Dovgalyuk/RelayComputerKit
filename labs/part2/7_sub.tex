\chapter{Арифметические операции. Вычитание}

\section{Вычитание двоичных чисел}

Вычитание в компьютерах делается с помощью трюков с переполнением.
Что такое $10-7=3$ в нашем четырёхбитном вычислителе?
Это $1010-0111=0011$. Так как разрядов всего $4$, тот же результат
можно получить с помощью сложения.

Вспомним, как работает сумматор. На его выходе всегда четырёхбитное число
и бит переноса. Если увеличивать $1010$, то сначала получится $1011$,
потом $1100$, $1101$, $1110$, $1111$ и наконец $0000$.

То есть мы к $10$ прибавили $5$, получили $15$
($1010+0101=1111$). После прибавления ещё одной единицы должно было бы получиться
$16$ ($10000$), но так как значащих битов всего $4$, пятый бит результата
пропал (точнее, стал сигналом переноса) и осталось значение $0$ ($0000$).

Остаётся увеличить это число на единицу ещё три раза, чтобы получился
правильный ответ~$0011$. Итого, мы прибавили к $10$ число $9$ вместо вычитания $5$
и получили нужный результат.

Следовательно, вместо вычитания $7$ ($0111$) можно прибавлять $9$ ($1001$),
когда речь идёт о четырёхбитных вычислениях.

Такой же фокус можно проделать с любым числом. Быстрый алгоритм
(без циклического прибавления единицы, пока не получится как надо) работает так:
инвертируем все биты числа, а затем прибавляем единицу.

Такое представление отрицательных чисел, для работы с которыми можно использовать обычный сумматор
называется дополнительным (потому что прибавляли единицу) обратным (потому что инвертировали) кодом.

Например, для $0111$ сначала получится $1000$, а потом $1001$, что и требовалось.
Работает и в обратную сторону --- из $1001$ инверсией получаем $0110$,
а после добавления единицы будет $0111$.

Таким образом, для вычитания не нужно придумывать новую сложную схему, а достаточно
использовать дополнительный обратный код для второго операнда и уже готовую схему для сложения.

\section{Практикум}

\subsection{Вычитание через сложение с дополнительным обратным кодом}

Список модулей:
\begin{itemize}
    \item Модуль переключателей: $3$ штуки
    \item Сумматор: $1$ штука
    \item Регистровый модуль: $1$ штука
\end{itemize}

Собрать схему для сложения.
Для вычитания вам нужно вычитаемое представлять в дополнительном обратном
коде.

\begin{enumerate}
    \item Отключить все управляющие сигналы.
    \item Установить перемычку для подачи сигнала на $\textasciitilde CY$.
    \item Набрать на тумблерах первого операнда значение $1010$ (число $10$).
    \item Набрать на тумблерах второго операнда значение $1101$ (число $-3$).
    \item Подключить выходной регистр к шине $3$. Убедиться, что в него записалось значение $0111$ (число $7$).
\end{enumerate}

\subsection{Вычитание с помощью сумматора и инвертора}

Чтобы вручную не переводить второй операнд, а втором входе сумматора автоматически получался дополнительный
обратный код, сначала нужно преобразовать исходное число с помощью инвертора, а затем прибавить $1$,
включив вход переноса в сумматоре.

Список модулей:
\begin{itemize}
    \item Модуль переключателей: $3$ штуки
    \item Модуль унарных операций: $1$ штука
    \item Сумматор: $1$ штука
    \item Регистровый модуль: $1$ штука
\end{itemize}

Собрать схему для сложения, а затем добавить между тумблерами со вторым операндом
инвертор (выход NOT блока унарной логики):

\includegraphics[width=0.5\columnwidth]{photo/subtractor.jpg}

\begin{enumerate}
    \item Отключить все управляющие сигналы.
    \item Установить перемычку для подачи сигнала на $CY$.
    \item Набрать на тумблерах первого операнда значение $1010$ (число $10$).
    \item Набрать на тумблерах второго операнда значение $0011$ (число $3$).
    \item Подключить выходной регистр к шине $3$. Убедиться, что в него записалось значение $0111$ (число $7$).
\end{enumerate}

\section{Задачи}

\begin{enumerate}
    \item Выполните деление числа $14$ на $3$ с помощью последовательности вычитаний.
          Записывайте промежуточные результаты на бумаге.
    \item Сделайте то же самое, только добавьте к схеме два регистра, чтобы результат очередного
          вычитания снова подать на вход сумматора.
\end{enumerate}

